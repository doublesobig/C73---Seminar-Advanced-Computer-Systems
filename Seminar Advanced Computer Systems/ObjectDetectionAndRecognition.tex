\documentclass[openany]{scrreprt}
%fixing wrong pagenumbers, adding header
\usepackage[automark, autooneside=true,headsepline]{scrlayer-scrpage}
\clearpairofpagestyles
\ihead{\pagemark}
\ohead{\rightmark}
\ofoot*{\pagemark}

%include libs
\usepackage{graphicx} %for implementing pictures
\usepackage[dvipsnames]{xcolor} %for costum colors
\usepackage{colortbl} % for colored cells
\usepackage{booktabs} % horizontal lines in table region (looks better than normal \hline}
\usepackage{tocstyle} % personalize table of contents
\usetocstyle{allwithdot} % adds dots in toc
\usepackage{parskip}
%declare costumized colors
\renewcommand{\familydefault}{\sfdefault}
\definecolor{htwgreen}{HTML}{76B900}
\definecolor{htwgrey}{HTML}{AFAFAF}
%stuff for personalized captions, chapters e.g.
\addtokomafont{sectioning}{\color{htwgreen}}
\addtokomafont{disposition}{\rmfamily}
\setkomafont{chapter}{\large\bfseries}
\setkomafont{section}{\normalsize\bfseries}
\setkomafont{subsection}{\normalsize\bfseries\centering}
\setkomafont{chapterentry}{\usekomafont{chapter}}
\settocfeature[toc][0]{entryhook}{\usekomafont{chapter}}
\settocfeature[toc][1]{entryhook}{\usekomafont{section}}


\begin{document}
\begin{titlepage}
	\flushright
	\includegraphics[width=0.4\textwidth]{./data/CE_Logo.png}\par\vspace{0cm}
	\Large{\textbf{\textcolor{htwgrey}{C73 - Seminar Advanced Computer Systems}}}\\
	\Large{\textbf{\textcolor{htwgrey}{Object Detection and Recognition}}}\\
	\Large{\textcolor{htwgreen}{Technical Report}}\\
	\Large{\textcolor{htwgrey}{Computer Engineering}}\\
	\begin{figure}[b]
	\flushright
	Version 1 \ \\
	\today{}
	\end{figure}
\end{titlepage}

\tableofcontents

\newpage

%
%
%
%
%
%
%
%
%	\chapter sollte für Gruppenthemen bleiben
%	\section für die einzelnen Personen mit ihren Themen
%	\in technical reports von einzelnen Personen bleiben also \subsection \subsubsection und \paragraph für die Unterteilung
%	Quellengaben bisher nur mit Fußnoten innerhalb des Textes \footnote{Name, V. des Autors: Retrieved from <Datum> from \textit{<Titel> (Year)}: <In welchem Artikel/Zeitschrift/Buch/etc. erschienen?, wo erschienen?. Hereafter referred to as \textit{<Kurzreferenz für spätere Zitate}}
%	vollständiges Quellenverzeichnis dann an Ende des Reports wie das Glossary anlegen
%	bitte Zuordnung von Namen zu Thema 
%
%
%
%

\chapter{Use this for Chapters} % sets numbers for table of content

\section{Use this for Sections} % subdivision of chapter above

Simple text should remain black

\subsection{Use this for Subsections}

\subsubsection{Use this for Subsubsections}

\paragraph{Non numbered header}\ \\

\begin{itemize}
\item Use this for not numbered listings
	\begin{itemize}
	\item this is how to do indent points
	\end{itemize}
\end{itemize}

\begin{enumerate}
\item Use this for numbered listings
	\begin{enumerate}
	\item works in the exact same way
	\end{enumerate}
\end{enumerate}

\begin{enumerate}
\item You can even mix things up
	\begin{itemize}
	\item like this
	\item will work the other way around aswell
	\end{itemize}
\end{enumerate}

\ \\

This is a template for tables
\begin{table}[!h]
\begin{tabular}{@{}cll@{}}
\toprule
\multicolumn{1}{l}{}                    & \multicolumn{1}{c}{\cellcolor[HTML]{AFAFAF}\textbf{Column 1}} & \multicolumn{1}{c}{\cellcolor[HTML]{AFAFAF}\textbf{Column 2}} \\ \midrule
\cellcolor[HTML]{AFAFAF}\textbf{Line 1} & Text                                                          & Text                                                          \\
\cellcolor[HTML]{AFAFAF}\textbf{Line 2} & Text                                                          & Text                                                          \\ \bottomrule
\end{tabular}
\caption{Template}
\label{tbl:example-table}
\end{table}

\newpage
%h = here
%t = top
%b = bottom
% p = page (place it on a page containing only floats, such as figures and tables
% ! allows to ignore certain parameters of LaTeX for float placement
\begin{figure}[!h]
\centering
\includegraphics[width=0.4\textwidth]{./data/CE_Logo.png}
\caption{Figure Text}
\label{img:example-figure}
\end{figure}

\newpage

\chapter{Glossary}
\begin{description}
\item[Glossary] 
could be done this way
\end{description}

\end{document}